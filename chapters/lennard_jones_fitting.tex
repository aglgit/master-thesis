In order to demonstrate the construction of a neural network
potential, we use Tensorflow to reconstruct the Lennard-Jones
potential. The Lennard-Jones potential is the simplest
realistic molecular dynamics potential, and since it is a function
of only radial distance, symmetry functions are not required.
We can thus use neural networks to perform a simple regression
on a function accepting one input and producing one output.
This serves to illustrate some of the intuitions and problems
one runs into when using more complex methods such as 
atom-centered descriptors and gives a nice introduction into
modern Tensorflow. We will be using the Tensorflow 2.0
beta version recently released, since it introduces a wide array
of changes which will likely prove influential to the long-term
direction of the Tensorflow project.
