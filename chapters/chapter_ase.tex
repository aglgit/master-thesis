The Atomic Simulation environment (ASE\footnote{
\url{https://wiki.fysik.dtu.dk/ase/index.html}})
[cite](https://iopscience.iop.org/article/10.1088/1361-648X/aa680e)
is a software package written in Python for the purpose
of setting up, steering and analyzing atomistic simulations.
Python is an interpreted, high-level general purpose language,
with a powerful, consise syntax which allows one to perform
very complex tasks with few lines of code. Python can also
easily be extended and interfaced with fast and mature
libraries. The modular interface of Python makes ASE
easily extensible: in particular the calculator interface
for evaluating energies, forces and much more has been
implemented for software packages such as LAMMPS, VASP,
Quantum Espresso and many more.
The Atomic Simulation Environment is intended to be:

\begin{itemize}
    \item Easy to use
    \item Flexible
    \item Customizable
    \item Pythonic
    \item Open to participation
\end{itemize}

The real drawback of Python is that it is an interpreted language,
which results in slow execution. It can also be quite memory-intensive,
which makes pure Python unsuitable for large scale computations
and simulations. It is therefore common to write the
computationally demanding tasks in a lower level compiled language,
and build a Python interface for calling functions and classes.

\subsection{Installation}
ASE requires an installation of

\begin{itemize}
    \item Python 2.7, 3.4-3.6
    \item Numpy 1.9 or newer
    \item Scipy 0.14 or newer
\end{itemize}

This can be easily obtained through the Anaconda 
or Miniconda packages\footnote{\url{https://anaconda.org/}},
or follow the instructions on the Python website\footnote{
\url{https://www.python.org/}}.
Once you have the prerequisites ASE can be installed using pip:

\begin{lstlisting}[language=bash]
pip install ase
\end{lstlisting}

\subsection{Molecular Dynamics}
Here we will demonstrate how to setup a simple Argon crystal,
set the velocities and integrate the system using
the Velocity Verlet equations.
First we import some prerequisites and define the system:

\begin{lstlisting}[language=python,basicstyle=\small]
from ase.lattice.cubic import FaceCenteredCubic
from ase import units
from ase.md.velocitydistribution import MaxwellBoltzmannDistribution
from ase.md.verlet import VelocityVerlet

symbol = "Ar"
size = (3, 3, 3)
atoms = FaceCenteredCubic(symbol=symbol, size=size, pbc=True)
MaxwellBoltzmannDistribution(atoms, 300 * units.kB)
\end{lstlisting}

This defines a face-centered-cubic (FCC) crystal unit cell
with 4 atoms, and a system size of $3\times3\times3$ unit cells
for a total of $4\cdot3^3 = 108$ atoms with periodic boundary conditions.
