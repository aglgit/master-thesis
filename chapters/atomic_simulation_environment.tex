The Atomic Simulation environment (ASE\footnote{
\url{https://wiki.fysik.dtu.dk/ase/index.html}})
\parencite[Larsen et al.]{larsen2017atomic}
is a software package written in Python for the purpose
of setting up, steering and analyzing atomistic simulations.
Python is an interpreted, high-level general purpose language,
with a powerful, consise syntax which allows one to perform
very complex tasks with few lines of code. Python can also
easily be extended and interfaced with fast and mature
libraries. The modular interface of Python makes ASE
easily extensible; in particular the calculator interface
for evaluating energies, forces and much more has been
implemented for open- and closed-source software packages 
such as LAMMPS, VASP, Quantum Espresso and many more.
The Atomic Simulation Environment is intended to be:

\begin{itemize}
    \item Easy to use
    \item Flexible
    \item Customizable
    \item Pythonic
    \item Open to participation
\end{itemize}

The real drawback of Python is that it is an interpreted language,
which results in slow execution. It can also be quite memory-intensive,
which makes pure Python unsuitable for large scale computations
and simulations. It is therefore common to write the
computationally demanding tasks in a lower level compiled language,
and build a Python interface for calling functions and classes.

\subsection{Installation}
ASE requires an installation of

\begin{itemize}
    \item Python 2.7, 3.4-3.6
    \item Numpy 1.9 or newer
    \item Scipy 0.14 or newer
\end{itemize}

This can be easily obtained through the Anaconda 
or Miniconda packages\footnote{\url{https://anaconda.org/}},
or follow the instructions on the Python website\footnote{
\url{https://www.python.org/}}.
Once you have the prerequisites ASE can be installed using pip:

\begin{lstlisting}[language=bash]
pip install ase
\end{lstlisting}

\subsection{Molecular Dynamics}
Here we will demonstrate how to setup a simple Argon crystal,
set the velocities and integrate the system using
the Velocity Verlet equations.
First we import some prerequisites and define the system:

\begin{minted}{python}
from ase.lattice.cubic import FaceCenteredCubic
from ase import units
from ase.md.velocitydistribution import MaxwellBoltzmannDistribution
from ase.md.verlet import VelocityVerlet

symbol = "Ar"
size = (3, 3, 3)
atoms = FaceCenteredCubic(symbol=symbol, size=size, pbc=True)
MaxwellBoltzmannDistribution(atoms, 300 * units.kB)
\end{minted}

This defines a face-centered-cubic (FCC) crystal unit cell
with 4 atoms, and a system size of $3\times3\times3$ unit cells
for a total of $4\cdot3^3 = 108$ atoms with periodic boundary conditions.
We thereafter give the atoms velocities according to the
Maxwell-Boltzmann distribution such that the system temperature
is approximately 300 Kelvin.
\par
Next we give the atoms a Lennard-Jones \textit{calculator}
for calculating dynamics and create a Velocity Verlet \textit{integrator}
for evolving the atoms according to their forces.
Calculators will be discussed in the next section.

\begin{minted}{python}
calc = LennardJones(sigma=3.405, epsilon=1.0318e-2)
atoms.set_calculator(calc)
dyn = VelocityVerlet(atoms, 2 * units.fs)
\end{minted}

The parameters $\sigma$ and $\epsilon$ define the well known
length and energy scales for the Lennard-Jones potential.
The Velocity Verlet integrator is initialized with a timestep
of 2 femtoseconds, which strikes a balance between accuracy
and the timescale of the simulation.
Finally we run the system for 1000 steps:

\begin{minted}{python}
for i in range(100):
    dyn.run(10)
\end{minted}

Every 10 timesteps the system quantities can be printed, written to file
or put through some other form of analysis or post-processing.

\subsection{Calculators}
The calculator interface gives ASE an easy to use,
flexible and customizable way of computing dynamics,
and makes ASE viable for a wide range of electronic structure calculations.
For ASE a calculator is a black box that receives positions, atomic
numbers and so on and outputs energies, forces, stresses;
all that is required to perform atomistic simulations.
The basic interface is defined in the ASE source code as:

\begin{minted}{python}
import numpy as np

class Calculator:
    def get_potential_energy(self, atoms=None, 
                             force_consistent=False):
        return 0.0

    def get_forces(self, atoms):
        return np.zeros((len(atoms), 3))

    def get_stress(self, atoms):
        return np.zeros(6)

    def calculation_required(self, atoms, quantities):
        return False
\end{minted}

There is also an interface for DFT calculators, which also
requires the implementation of spins, occupation numbers,
the fermi level and so on.
\newline
\newline
In the previous section we used the Lennard-Jones calculator
as an example, however this calculator is not suited for general use.
Apart from the Lennard-Jones being a toy potential unsuited
for real systems, the ASE implementation is also written in pure
Python, which makes it very, very slow.
Usually the calculator interface is used as a wrapper for much faster
compiled code or software packages; for example the
Asap calculator is a Python wrapper for the Effective Medium Theory
potential, which is implemented in Fortran and is orders of magnitudes
faster. A Python molecular dynamics code is justified
since the bulk of computation time is spent on computing forces,
however for large scale molecular dynamics simulations
one would be advised to use fast compiled code such as LAMMPS\footnote{
\url{https://lammps.sandia.gov/}} or GROMACS\footnote{
\url{http://www.gromacs.org/}}.
ASE really shines when it comes to small-scale simulations,
experimentation and testing out new methods, and this is why
it has been chosen as a basis for this thesis.
Fortunately, ASE has an extensive code base of calculators
for Molecular Dynamics and Density Functional Theory codes
such as Asap, CP2K, LAMMPS, GROMACS which are all
implemented in lower-level languages.
