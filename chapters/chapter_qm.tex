The discussion in this chapter follows closely the discussion in
[Sakurai | Modern Quantum Mechanics].
\newline
In quantum mechanics, a physical state is represented by a textit{state vector}
in a complex vector space. Such a vector is called a \textit{ket}, denoted
by $\ket{\alpha}$. The state ket is postulated to contain all information
about the physical state. Two kets can be added to produce a new ket:
$$ \ket{\alpha} + \ket{\beta} = \ket{\gamma} .$$
They can also be multiplied by a complex number:
$$ c\ket{\alpha} = \ket{\alpha}c = \ket{\delta} .$$
If $c$ is zero the resulting ket is called a \textit{null ket}.
If $c$ is non-zero it is postulated that the resulting ket contains
the same information.
\newline
Observables such as momentum and spin are represented by operators
acting on the vector space in question. Operators
act on a ket from the left to produce a new ket:
$$ A \ket{\alpha} = \ket{\delta} .$$
Of particular importance is when the action of an operator
on a ket is the same multiplication:
$$ A \ket{\alpha} = c\ket{\alpha} = \ket{\delta} .$$
These kets are known as \textit{eigenkets} and the corresponding
complex numbers are known as \textit{eigenvalues}.
The physical state represented by an eigenket is known
as an \textit{eigenstate}. Any ket
can be written as an expansion of eigenkets $\ket{a'}$:
$$ \ket{\alpha} = \sum_{a'} c_{a'} \ket{a'} ,$$
where $c_{a'}$ is a complex coefficient. In principle
there are infinitely many linearly indepedent eigenkets,
depending on the dimensionality of the vector space.
The uniqueness of the expansion can be proven
with orthonogality of the eigenkets, which we will simply postulate.
\par
A \textit{bra space} is a vector space "dual" to the ket space.
We postulate that for every ket $\ket{\alpha}$ there exists a bra
$\bra{\alpha}$. The bra space is spanned by eigenbras $\bra{a'}$
corresponding to the eigenkets $\ket{a'}$. The ket and bra spaces
have a dual correspondence:
