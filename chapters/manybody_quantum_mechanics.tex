The Schr\"{o}dinger equation only offers exact solutions
for very small and simple systems, such as the hydrogen atom
or a system of non-interacting harmonic oscillators.
If we want to study most systems of interest we must turn
instead to numerical methods and solutions.
The most prominent methods in the field of manybody
quantum mechanics are the Hartree-Fock (HF) and the
Density Functional Theory (DFT) families of solvers.
This chapter will give a brief overview of both methods,
focused more on understanding and intuition than rigor.
Eventually we will take these methods to be black boxes of
electronic structure calculations, given some cartesian
coordinates and outputting energies and forces.
This section will give a brief overview of the discussion
of the electronic Hamiltonian in \parencite[Szabo][pages 39-89]
{szabo1996modern}, which also covers the Hartree-Fock theory
which will be expanded upon in the next section.
\par
We want to find solutions to the non-relativistic time-independent
Schr\"{o}dinger equation:

\begin{equation}
 \hat{H} \ket{\Psi} = E \ket{\Psi} ,
\end{equation}

with the Hamiltonian $\hat{H}$ describing a system of nuclei and electrons
with cartesian coordinates $\bm{R}_a, \ a=1,2,\dots,A$ and $\bm{r}_i, \
i=1,2,\dots,N$ respectively. The distance between nuclei $a$
and electron $i$ is given as the euclidean distance
$R_{ai} = \left| \bm{R}_a - \bm{r}_i \right|$
and correspondingly for the nuclei-nuclei and electron-electron distances.
The full Hamiltonian for a set of $N$ electrons and $A$ nuclei
in atomic units is:

\begin{equation}
    \begin{split}
        \hat{H}
        &= -\sum_{i=1}^N \frac{1}{2} \nabla_i^2
        -\sum_{a=1}^A \frac{1}{M_a} \nabla_a^2
        -\sum_{i=1}^N \sum_{a=1}^A \frac{Z_a}{R_{ia}} \\
        &+ \sum_{i=1}^N \sum_{j=i+1}^N \frac{1}{r_{ij}}
        + \sum_{a=1}^A \sum_{b=a+1}^A \frac{Z_a Z_b}{R_{ab}}
    \end{split} .
\end{equation}

The first two terms describe the kinetic energy operators
of the electrons and nuclei, with $M_a$ the ratio of the mass
of nuclei $a$ to the electron mass. The third term describes the
coulomb attraction between electrons and nuclei, while the fourth and fifth
terms describe the repulsion between electrons and nuclei respectively.
\par
Since the nuclei are approximately 2000 times heavier
than the electrons, the electrons can to a good approximation
be described as moving in the field of fixed nuclei. In practice we
neglect the kinetic energy terms of the nuclei, while considering
an averaged effect from the nuclei-nuclei repulsion.
The nuclei-nuclei repulsion energy averaged over 
adds a constant to the energy
eigenvalues, but has no effect on the energy eigenfunctions.
The remaining terms are known as the electronic Hamiltonian:

\begin{equation}
    \hat{H}_e = -\sum_{i=1}^N \frac{1}{2} \nabla_i^2
    -\sum_{i=1}^N \sum_{a=1}^A \frac{Z_A}{R_{ia}}
    +\sum_{i=1}^N \sum_{j=i+1}^N \frac{1}{r_{ij}} .
\end{equation}

The electronic wavefunction $\Psi_e = \Psi_e(\{r_i\}; \{R_a\})$
is a function of the electronic coordinates with a parametric dependence
on the fixed nucleic coordinates. The electronic energy
is obtained in the usual way $E_{e} = \braket{\Psi_e | \hat{H}_e |
\Psi_e} $. The total energy of our system
must now include the constant nuclear repulsion:

\begin{equation}
 E_{tot} = E_{e} + \sum_{a=1}^A \sum_{b=a+1}^A
    \frac{Z_a Z_b}{R_{ab}} . 
\end{equation}

If one has solved the Schr\"{o}dinger equation for the electronic
Hamiltonian, one can subsequently solve for the nucleic motion
using the same trick, i.e. substituting the electronic coordinates
for their average values, averaged over the electronic wave function.
We are then left with a nuclear Hamiltonian $\hat{H}_n$:

\begin{equation}
    \begin{split}
        \hat{H}_n
        &= -\sum_{a=1}^A \frac{1}{2 M_a} \nabla_a^2
        + \biggl< -\sum_{i=1}^N \frac{1}{2} \nabla_i^2
        - \sum_{i=1}^N \sum_{a=1}^A \frac{Z_a}{R_{ia}}
        + \sum_{i=1}^N \sum_{j=i+1}^N \frac{1}{r_{ij}}
        \biggr> \\
        &+ \sum_{a=1}^A \sum_{b=a+1}^A
        \frac{Z_a Z_b}{R_{ab}} \\
        &= -\sum_{a=1}^A \frac{1}{2 M_a} \nabla_a^2
        + E_{tot} .
    \end{split}
\end{equation}

Under this approximation the nuclei move on a potential energy
surface obtained by solving the electronic Hamiltonian.
We will however remain focused on the electronic structure problem.
For the electronic Hamiltonian we would like to make one more
approximation.
The electronic Hamiltonian further simplifies if instead
of considering the relative positions of the nuclei,
we fix the coordinate system in the center of mass of the nucleus
and neglect the coordinates of individual nucleic constituents.
This gives us the final expression we want for the electronic
Hamiltonian:

\begin{equation}
    \hat{H} = -\sum_{i=1}^N \frac{1}{2} \nabla_i^2
    - \sum_{i=1}^N \frac{Z}{r_{i}} + \sum_{i=1}^N \sum_{j=i+1}^N
    \frac{1}{r_{ij}}
\end{equation}

\subsection{Hartree-Fock}
The Hartree-Fock method is a method for finding
solutions to the electronic Hamiltonian assuming
the electron-electron repulsion can be approximated
with a set of single-particle functions or \textit{orbitals},
moving in a mean field generated by the presence of other electrons.
The theory in this section is based upon the \parencite[Sherrill]
{sherrill2000} lecture notes from the Georgia Institute of Technology
and the lecture notes
on Computational Physics II \parencite[Hjorth-Jensen]
{hjensen2019} from the University of Oslo.
Assuming that the electrons do not interact
the Hamiltonian is separable and the wavefunction
is simply a product of orbitals $\psi$
which are solutions to a onebody Hamiltonian.
This gives us an ansatz for the manybody wavefunction $\Psi$
known as the \textit{Hartree product}:

\begin{equation}
 \Psi(\bm{r}_1,\dots,\bm{r}_N) = \psi(\bm{r}_1) \cdot \dots
    \cdot \psi(\bm{r}_N) . 
\end{equation}

Since we are dealing with fermions this ansatz fails to satisfy
the antisymmetry principle, i.e. the wavefunction
is not antisymmetric with respect to the interchange of any two
particles. Fermions in addition to three spatial degrees of freedom
also have a spin degree of freedom $\sigma$
which means the fermion can be described
by the space-spin coordinate $\bm{x} = (\bm{r}, \sigma)$
with $\bm{x} \in \mathbb{R}^3 \otimes \sigma$.
The problem of antisymmetry in a system of $N$ fermions
is satisfied by the introduction of \textit{Slater determinants}

\begin{equation}
\Psi(\bm{x}_1,\dots,\bm{x}_N)
= \frac{1}{\sqrt{N!}}
\begin{vmatrix}
    \chi_{1}(\bm{x}_1) & \chi_{2}(\bm{x}_1)
    & \dots & \chi_{N}(\bm{x}_1) \\
    \chi_{1}(\bm{x}_2)  & \chi_{2}(\bm{x}_2)
    & \dots & \chi_{N}(\bm{x}_2) \\
    \vdots & \vdots & \ddots & \vdots \\
    \chi_{1}(\bm{x}_N) & \chi_{2}(\bm{x}_N)
    & \dots & \chi_{N}(\bm{x}_N)
\end{vmatrix} ,
\end{equation}

with $\chi(\bm{x})$ spin orbitals and a normalization factor
$(N!)^{-1/2}$. The introduction of this ansatz is equivalent to assuming that
all electrons move independently of each other
in a mean field generated by the electron-electron repulsion.
Define the one-electron operator of the electronic Hamiltonian as:

\begin{equation}
 \hat{h}_1(\bm{x}_i) = -\frac{1}{2} \nabla_i^2
    -\frac{Z}{r_i} , 
\end{equation}

with a twobody interaction term

\begin{equation}
 \hat{v}(\bm{x}_i, \bm{x}_j) = \frac{1}{r_{ij}} , 
\end{equation}

which allows us to write the electronic Hamiltonian more compactly as:

\begin{equation}
 \hat{H} = \sum_i \hat{h}_1(\bm{x}_i)
    + \sum_{i < j} \hat{v}(\bm{x}_i, \bm{x}_j) .
\end{equation}

The expectation value of the energy is given as the usual inner product:

\begin{equation}
 E = \braket{\Psi | \hat{H} | \Psi} .
\end{equation}

The \textit{variational theorem} states that the expectation
value of any normalized wavefunction with respect to the energy
represents an upper bound to the ground state energy.
This suggests a procedure wherein we vary the parameters
of a set of approximate wave functions $\Psi_T$
until an energy minimum is reached.
The Hartree-Fock energy can be written in terms of integrals
over the onebody and interaction terms:

\begin{equation}
    E_{HF} = \braket{\Psi | \hat{H} | \Psi} = \sum_i \braket{i | \hat{h} | i}
    + \sum_{i < j} \braket{ij | \hat{v} | ij}_{AS} ,
\end{equation}

where we have introduced an antisymmetrized matrix element:

\begin{equation}
 \braket{ij | \hat{v} | ij}_{AS}
    = \braket{ij | \hat{v} | ij} - \braket{ij | \hat{v} | ji} , 
\end{equation}

and the shorthand integrals:

\begin{equation}
 \braket{i | \hat{h}_1 | i} =
    \int d\bm{r} \chi_i^*(\bm{r}) \hat{h}_1 \chi_i(\bm{r}) , 
\end{equation}

and

\begin{equation}
 \braket{ij | \hat{v} | ij} =
    \int d\bm{r}_i d\bm{r}_j \chi_i^*(\bm{r}_i) \chi_j^*(\bm{r}_j) 
    \hat{v} \chi_i(\bm{r}_i) \chi_j(\bm{r}_j) . 
\end{equation}

In order to derive the Hartree-Fock equations
we perform a linear expansion of the spin orbitals
$\chi$ in terms of a fixed orthogonal basis $\phi$:

\begin{equation}
 \chi_i = \sum_{\nu} C_{i\nu} \phi_{\nu} , 
\end{equation}

in principle an infinite sum, but in practice truncated.
This basis is usually obtained as the eigenfunctions of
parts of the electronic Hamiltonian, e.g. solutions
to the Schr\"{o}dinger equation with a harmonic oscillator
or Coulomb potential.
If the coefficients belong to an orthogonal or unitary matrix,
the resulting basis will preserve orthogonality.
This expansion allows us to rewrite the Hartree-Fock energy as:

\begin{equation}
 E_{HF} = \sum_i \sum_{\alpha \beta}
    C_{i \alpha}^* C_{i\beta} \braket{\alpha | \hat{h}_1 | \beta}
    + \sum_{i < j} \sum_{\alpha \beta \delta \eta}
    C_{i\alpha}^* C_{j\beta}^* C_{i\delta} C_{j\eta}
    \braket{\alpha \beta | \hat{v} | \delta \eta} .
\end{equation}

Using the method of Lagrangian multipliers
we can define a functional to be minimized:

\begin{equation}
\begin{split}
F[\Psi] &= E_{HF}[\Psi] - \sum_i \epsilon_i \braket{i | j} \\
        &= E_{HF}[\Psi] - \sum_i \epsilon_i \sum_{\alpha}
        C_{i\alpha}^* C_{i\alpha} ,
\end{split}
\end{equation}

with Lagrange multipliers $\epsilon_i$, where we have
exploited the orthogonality of the basis functions to
introduce the coefficients. The Lagrange multipliers
are identified with the energy eigenvalues
of the single-particle orbitals.
Minimizing with respect to $C_{i\alpha}^*$ yields
the eigenvalue equation:

\begin{equation}
\sum_{\beta} \hat{h}_{\alpha\beta}^{HF} C_{i\beta}
= \epsilon_i C_{i\alpha} ,
\end{equation}

where we have introduced the Fock matrix elements:

\begin{equation}
\hat{h}_{\alpha\beta}^{HF} = 
\braket{\alpha | \hat{h}_1 | \beta}
+ \sum_{j}^N \sum_{\delta\eta} C_{j\delta}^* C_{j\eta}
\braket{\alpha\delta | \hat{v} | \beta\eta}_{AS} .
\end{equation}

The single-particle integrals are usually tabulated
in advance, and depend upon the choice of basis functions.
Often the single-particle integrals $\braket{\alpha | \hat{h}_1 | \beta}$
have analytical expressions, while the antisymmetric matrix elements
must be evaluated using numerical integration.
The eigenvalue problem can be written more compactly as:

\begin{equation}
FC = C\epsilon ,
\end{equation}

where $F$ is the Fock matrix defined above, $C$ is the matrix
of coefficients and $\epsilon$ is now
a vector of single-particle energies.
The Hartree-Fock equations are solved in an iterative way,
starting with an initial guess for the coefficients $C$.
Solving the eigenvalue problem yields new eigenvectors
and eigenvalues.
The process continues until the change in eigenvalues
is within some tolerance $\nu$:

\begin{equation}
\frac{\sum_p \left| \epsilon_i^n - \epsilon_i^{n-1} \right|}
{m} \leq \nu ,
\end{equation}

where $p$ runs over all the single-particle energies and
$m$ is the number of single-particle states.

\subsection{Density-functional theory}
Density-functional theory (DFT) is a method for
investigating the electronic structure of a manybody system
by finding approximations to the ground state
density $n(\bm{r})$.
It holds many similarities to the Hartree-Fock method,
since the usual method of obtaining the ground-state density
involves constructing a single-determinant wave function
from a set of orthonormal single-particle states,
and expanding these single-particle states in terms
of a known basis. 
However, DFT methods offer the explicit treatment of both the
electron exchange and electron correlation interactions,
while Hartree-fock only includes
the exchange energies.
\par
This section is a brief summary of the material covered
in the \parencite[Toulouse][pages 1-12]{toulouse2017}
lecture notes from the Université Pierre et Marie Curie,
which the reader is encouraged to check out for more detail.
Our starting point is again the electronic Hamiltonian:

\begin{equation}
    \hat{H} = -\sum_{i=1}^N \frac{1}{2} \nabla_i^2
    - \sum_{i=1}^N \frac{Z}{r_{i}} + \sum_{i=1}^N \sum_{j=i+1}^N
    \frac{1}{r_{ij}} .
\end{equation}

Any electronic wavefunction $\Psi$ which solves this equation
is in principle a function of $4N$ coordinates $\bm{x}_i = (\bm{r}_i, \sigma_i)
, \ i=1,\dots,N$
where $N$ is the number of electrons.
Once we have obtained a solution to the Schr\"{o}dinger equation
we can obtain the one-electron density $n(\bm{r})$ as:

\begin{equation}
    n(\bm{r}) = N \int \left| \Psi(\bm{r}, \sigma, \bm{x}_2,\dots,\bm{x}_N)
    \right|^2 d\sigma d\bm{x}_2 \dots d\bm{x}_N .
\end{equation}

Since the wavefunction is a unique functional of the Hamiltonian
$\hat{H}$, the one-electron density is uniquely determined
by the Hamiltonian. Hohenberg and Kohn showed in 1964
\cite{hohenberg1964inhomogeneous} in their first theorem that this mapping
can be inverted, i.e. that the one-electron density uniquely
determines the Hamiltonian of our system (up to an arbitrary constant).
Taken altogether, this means that all properties of our system, including
the Hamiltonian and the manybody wavefunction are fixed
by a one-electron density carrying a dependency on only 3 spatial coordinates.
The electronic Hamiltonian can be rewritten as:

\begin{equation}
 \hat{H} = \hat{F} + \hat{V}_{ne} , 
\end{equation}

where $\hat{F}$ is an operator consisting of the kinetic energy
and electron-electron operators and $\hat{V}_{ne}$
is the electron-nuclei interaction.
For their second theorem, Hohenberg and Kohn defined
the universal density-functional:

\begin{equation}
 F[n] = \braket{\Psi[n] | \hat{F} | \Psi[n]} , 
\end{equation}

and the total electronic energy functional:

\begin{equation}
 E[n] = F[n] + \int \hat{V}_{ne} n(\bm{r}) d\bm{r} . 
\end{equation}

Hohenberg and Kohn showed that the energy functional
with respect to one-electron densitites $n(\bm{r})$
is an upper bound to the ground state energy:

\begin{equation}
 E_0 \leq F[n] + \int \hat{V}_{ne} n(\bm{r}) d\bm{r} , 
\end{equation}

with equality if and only if the one-electron density
is the one-electron density corresponding to the Hamiltonian $\hat{H}$.
This suggests a variational procedure, wherein
we vary the total electronic energy functional
until we reach an energy minimum:

\begin{equation}
 E_{min} = \underset{n}{\min} \ E[n] , 
\end{equation}

which serves as our best estimate for the ground state
one-electron density $n(\bm{r})$.
Levy and Lieb \cite{levy1979universal, perdew1982density, perdew1983physical}
proposed to redefine the universal density-functional
in terms of normalized antisymmetric wavefunctions $\Psi$
which yield a fixed density $n$:

\begin{equation}
 F[n] = \underset{\Psi \rightarrow n}{\min}
    \braket{\Psi | \hat{F} | \Psi}
    = \braket{\Psi[n] | \hat{F} | \Psi[n]} , 
\end{equation}

wherein the minima search is performed over wavefunctions
which yield the fixed density $n$.
A search is then performed over densitities $n$
until we reach an energy minimum. This method
is known as the \textit{constrained search formulation}.
Kohn and Sham \cite{PhysRev.140.A1133} proposed to decompose $F[n]$ as:

\begin{equation}
 F[n] = T_s[n] + E_{\text{Hxc}}[n] , 
\end{equation}

where $T_s[n]$ is a non-interacting kinetic-energy functional
which can be defined through the constrained-search formulation:

\begin{equation}
 T_s[n] = \underset{\Phi \rightarrow n}{\min}
    \braket{\Phi | \hat{T} | \Phi}
    = \braket{\Phi[n] | \hat{T} | \Phi[n]} , 
\end{equation}

wherein the minima search is now performed over normalized
single-determinant wavefunctions $\Phi$ which yield
the fixed density $n$. The functional
$E_{\text{Hxc}}[n]$ is known as the Hartree-exchange-correlation
functional. The variational procedure is now
performed over single-determinant wavefunctions
which yield a fixed density $n$ and then
minimized over densities:

\begin{equation}
    \begin{split}
        E_0
        &= \underset{n}{\min} \left\{
            F[n] + \int \hat{V}_{ne} n(\bm{r}) d\bm{r} \right\} \\
        &= \underset{n}{\min} \left\{
            \underset{\Phi \rightarrow n}{\min}
            \braket{\Phi | \hat{T} | \Phi}
            + E_{\text{Hxc}}[n] + \int \hat{V}_{ne} n(\bm{r}) d\bm{r} \right\} \\
        &= \underset{n}{\min} \ \underset{\Phi \rightarrow n}{\min}
        \left\{ \braket{\Phi | \hat{T} + \hat{V}_{ne} | \Phi}
        + E_{\text{Hxc}}[n_{\Phi}]
        \right\} \\
        &=  \underset{\Phi}{\min}
            \left\{ \braket{\Phi|\hat{T} + \hat{V}_{ne}|\Phi}
            + E_{\text{Hxc}}[n_{\Phi}]\right\}.
    \end{split}
\end{equation}

These equations now involve only a single-determinant wave function,
which is a large simplification over a variational method involving 
multi-determinant wave functions.
Now a major part of the kinetic energy contribution
can be treated through the single-determinant wave function,
while only the Hartree-exchange-correlation needs to be approximated
as a functional of the density.
As with the Hartree-Fock method, the single determinant
wavefunctions are constructed from an orthonormal basis of spin orbitals
$\chi_i (\bm{x}), \ i=1,\dots,M$ with spin-spatial
coordinates $\bm{x}_i = (\bm{r}_i, \sigma_i)$.
The total electronic energy can be expressed in terms of spatial
orbitals $\phi_i(\bm{r})$ after integrating out the spin variables:

\begin{equation}
    E[\left\{\phi_i\right\}] 
    = \sum_i \int \phi_i^* (\bm{r}) (-\frac{1}{2} \nabla^2 + \hat{V}_{ne})
        \phi_i(\bm{r}) d\bm{r} + E_{\text{Hxc}}[n] ,
\end{equation}

with the density expressed as:

\begin{equation}
    n(\bm{r}) = \sum_i \left| \phi_i(\bm{r}) \right|^2 .
\end{equation}

The energy minimum is obtained using the method of Lagrangian multipliers,
with the constraint that the spatial orbitals be normalized
we introduce the following Lagrangian:

\begin{equation}
 \mathcal{L} [\left\{\phi_i\right\}]
    = E[\left\{\phi_i\right\}] - \sum_i \epsilon_i
    \left( \int \phi_i^* (\bm{r}) \phi_i(\bm{r}) d\bm{r} - 1 \right) , 
\end{equation}

with $\epsilon_i$ the associated Lagrangian multiplier.
The energy minimum is where the Lagrangian is stationary
with respect to the spatial orbitals:

\begin{equation}
 \frac{\partial \mathcal{L}}{\partial \phi_i^* (\bm{r})} = 0 , 
\end{equation}

which gives us the functional derivative:

\begin{equation}
 \left( -\frac{1}{2} \nabla^2 + \hat{V}_{ne} \right) \phi_i(\bm{r})
    + \frac{\partial E_{\text{Hxc}}[n]}{\partial \phi_i^*} 
    = \epsilon_i \phi_i(\bm{r}) . 
\end{equation}

The second term can be expressed through the chain rule as:

\begin{equation}
 \frac{\partial E_{\text{Hxc}}[n]}{\partial \phi_i^*} =
    \int \frac{\partial E_{\text{Hxc}}}{\partial n(\bm{r}')}
    \frac{\partial n(\bm{r}')}{\partial \phi_i^*(\bm{r})} d\bm{r}' .
\end{equation}

The second factor can be expressed as:

\begin{equation}
 \frac{\partial n(\bm{r}')}{\partial \phi_i^*(\bm{r})}
    = \phi_i(\bm{r}) \delta (\bm{r} - \bm{r}') .
\end{equation}

Defining the Hartree-exchange-correlation potential as the functional
derivative:

\begin{equation}
 \hat{V}_{\text{Hxc}} = \frac{\partial E_{\text{Hxc}}}{\partial
    \phi_i^*(\bm{r})} , 
\end{equation}

we arrive at the Kohn-Sham eigenvalue equations:

\begin{equation}
 \left( -\frac{1}{2} \nabla^2 + \hat{V}_{ne} + \hat{V}_{\text{Hxc}}
    \right) \phi_i(\bm{r}) = \epsilon_i \phi_i (\bm{r}) . 
\end{equation}

The eigenfunctions satisfying these equations are known as the
Kohn-Sham orbitals, and are eigenfunctions of the Kohn-Sham
one-electron Hamiltonian:

\begin{equation}
 \hat{h}_{KS} = - \frac{1}{2} \nabla^2 + \hat{V}_{KS} , 
\end{equation}

with the Kohn-Sham potential defined as:

\begin{equation}
 \hat{V}_{KS} = \hat{V}_{ne} + \hat{V}_{\text{Hxc}} . 
\end{equation}

The Kohn-Sham one-electron Hamiltonian defines a system of $N$
non-interacting electrons in an effective external potential
$\hat{V}_{KS}$ ensuring that the one-electron density
is the same as that of the ground-state one-electron density
of a system of interacting electrons.
The Hartree-exchange-correlation potential is further
decomposed as:

\begin{equation}
\hat{V}_{\text{Hxc}} = \hat{V}_{\text{H}} + \hat{V}_{\text{XC}} ,
\end{equation}

where $\hat{V}_{\text{H}}$ is the Hartree potential and
$\hat{V}_{\text{XC}}$ is the exchange-correlation potential.
The Hartree potential can be expressed through the density:

\begin{equation}
\hat{V}_{\text{H}} = \int \frac{n(\bm{r})}{\left| \bm{r} -
    \bm{r}' \right|} d\bm{r}' .
\end{equation}

The exchange-correlation potential is decomposed
into the exchange potential and the correlation potential:

\begin{equation}
\hat{V}_{\text{XC}} = \hat{V}_{\text{X}} + \hat{V}_{\text{C}} .
\end{equation}

For practical calculation the 
spatial orbitals are expanded as a linear combination
of a known basis, such as hydrogen-like functions
or Gaussian-type orbitals:

\begin{equation}
 \phi_i(\bm{r}) = \sum_{\nu} C_{\nu i} \chi_i (\bm{r}) . 
\end{equation}

We insert these into the Kohn-Sham equations, multiply
by $\chi_i^*(\bm{r})$ and integrate over $\bm{r}$ to arrive
at the eigenvalue equations:

\begin{equation}
 \sum_{\nu} F_{\mu\nu} C_{\nu i} =
    \epsilon_i \sum_{\nu} S_{\mu\nu} C_{\nu i} , 
\end{equation}

where $F_{\mu\nu} = \int \chi_{\mu}^*(\bm{r}) \hat{h}_{KS}
\chi_{\nu}^*(\bm{r}) d\bm{r} $ are the elements of the Kohn-Sham
Fock matrix and $S_{\mu\nu} = \int \chi_{\mu}^*(\bm{r}) \chi_{\nu}
(\bm{r}) d\bm{r}$ are the elements of the overlap matrix
of basis elements.
The Fock matrix is decomposed into its constituent parts:

\begin{equation}
F_{\mu\nu} = H_{\mu\nu} \ J_{\mu\nu} + V_{XC,\mu\nu} ,
\end{equation}

where $H_{\mu\nu}$ are the one-electron integrals:

\begin{equation}
H_{\mu\nu} = \int \chi_{\mu}^*(\bm{r})
    \left( -\frac{1}{2} \nabla^2 + \hat{V}_{ne}(\bm{r}) \right)
    \chi_{\mu}(\bm{r}) d\bm{r} ,
\end{equation}

$J_{\mu\nu}$ is the Hartree-potential contribution:

\begin{equation}
    J_{\mu\nu} = \sum_{\lambda} \sum_{\gamma} P_{\lambda\gamma}
    \left(\chi_{\mu}\chi_{\nu} | \chi_{\lambda}\chi_{\gamma}\right) ,
\end{equation}

where we have defined the density matrix $P_{\lambda\gamma}$ as:

\begin{equation}
P_{\gamma\lambda} = \sum_i C_{\gamma i} C_{\lambda i}^* ,
\end{equation}

and the two-electron integrals are defined as:

\begin{equation}
\left(\chi_{\mu}\chi_{\nu} | \chi_{\lambda}\chi_{\gamma}\right)
    = \int \int \frac{\chi_{\mu}^*(\bm{r}_1) \chi_{\nu}
    \chi_{\lambda}^*(\bm{r}_2) \chi_{\gamma}}
    {\left| \bm{r}_1 - \bm{r}_2 \right|}
    d\bm{r}_1 d\bm{r}_2 .
\end{equation}

and finally the exchange-correlation potential contribution:

\begin{equation}
\hat{V}_{\text{XC},\mu\nu} = \int \chi_{\mu}^*(\bm{r})
    \hat{V}_{\text{XC}}(\bm{r}) \chi_{\mu}(\bm{r}) d\bm{r} .
\end{equation}

The Kohn-Sham equations can be written more compactly as the matrix equation:

\begin{equation}
 FC = SC\epsilon , 
\end{equation}

where $\epsilon$ is a vector containing the energy eigenvalues
of the basis elements. The equations are then solved
iteratively through diagonalization to obtain
the matrix of coefficients $C$. Once we have obtained the 
coefficients the density $n(\bm{r})$ is calculated as:

\begin{equation}
 n(\bm{r}) = \sum_{\gamma} \sum_{\lambda} P_{\gamma\lambda}
    \chi_{\gamma}(\bm{r}) \chi_{\lambda}^*(\bm{r}) , 
\end{equation}

where the summation is over the number of basis functions $M$.
The total electronic energy can now be expressed
as:

\begin{equation}
E = \sum_{\mu} \sum_{\nu} P_{\nu\mu} H_{\mu\nu}
    +\frac{1}{2} \sum_{\mu} \sum_{\nu} P_{\nu\mu}
    J_{\mu\nu}
    + E_{\text{XC}} .
\end{equation}

Typically the energy contribution from the exchange-correlation
$E_{\text{XC}}$ has a complicated non-linear dependence on the
density and must therefore be evaluated through
numerical integration.
