Our starting point for molecular dynamics is
the full Hamiltonian for a set of $N$ electrons and $A$ nuclei:

\begin{equation}
    \begin{split}
        \hat{H} 
        &= -\sum_{i=1}^N \frac{1}{2} \nabla_i^2
        -\sum_{a=1}^A \frac{1}{M_a} \nabla_a^2
        -\sum_{i=1}^N \sum_{a=1}^A \frac{Z_a}{r_{ia}} \\
        &+ \sum_{i=1}^N \sum_{j=i+1}^N \frac{1}{r_{ij}}
        + \sum_{a=1}^A \sum_{b=a+1}^A \frac{Z_a Z_b}{R_{ab}}
    \end{split} .
\end{equation}

We want to find solutions to the time-dependent non-relativistic
Schrodinger equation:

$$ i\hbar \frac{\partial}{\partial t} \Psi = \hat{H} \Psi . $$

\subsection{From quantum mechanics to molecular dynamics}

We will follow the route of Tully described in
[https://core.ac.uk/download/pdf/35009882.pdf?repositoryId=810](here).
The wavefunction is separated in terms of the electronic and nuclear
coordinates with an ansatz

$$ \Psi(\set{\bm{r}_i}, \set{\bm{R}_I}, t)
    \approx \Psi(\set{\bm{r}_i}) \chi(\set{\bm{R}_I})
    \exp \left[ \frac{i}{\hbar} \int_{t_0}^t
    dt^{'} \hat{E}_e(t^{'}) \right] ,
$$

with the electronic and nuclear wavefunctions normalized to unity
at every instance of time. A phase factor is introduced to make
the equations look nice:

$$ \hat{E}_e = \int d\bm{r} d\bm{R} \Psi* \chi* \hat{H} \Psi \chi , $$

where the integration occurs over all spatial coordinates
$\set{\bm{r}_i}, \set{\bm{R}_I}$. This is a single determinant
ansatz which must lead to a mean-field description of the dynamics.
This ansatz differs from the Born-Oppenheimer approximation:

$$ \Psi_{BO} = \sum_{k=0}^{\infty} \Psi_k \chi_k , $$

even in the single-determinant limit where only
a single state $k$ is included in the expansion.
\par
Inserting this ansatz into the Schrodinger equation
reveals the following set of equations:

$$ i\hbar \frac{\partial \Psi}{\partial t} 
    = -\sum_i \frac{\hbar^2}{2m_e} \nabla_i^2 \Psi
    + \set{ \int d\bm{R} \chi^* V_{ne} \chi } \Psi , $$

$$ i\hbar \frac{\partial \chi}{\partial t} 
    = -\sum_i \frac{\hbar^2}{2m_e} \nabla_I^2 \chi
    + \set{ \int d\bm{r} \Psi^* \hat{H} \Psi } \chi . $$

This set of equations is the basis for the time-dependent
self-consistent field (TDSCF) method, wherein particles move
in time-dependent effective potentials obtained from quantum
mechanical expectation values.
\par
In the framework of classical molecular dynamics
we approximate the nuclei as classical point particles.
This can be done be rewriting the nuclear wavefunction as

$$ \chi = A \exp[iS/\hbar] , $$

with an amplitude factor $A$ and a phase $S$
which are both considered to be real.
The TDSCF equations are rewritten in terms of these variables

$$ \frac{\partial S}{\partial t} + \sum_I \frac{1}{2M_I}
    (\nabla_I S)^2 + \int d\bm{r} \Psi^* \hat{H} \Psi
    = \hbar^2 \sum_I \frac{1}{2M_I} \frac{\nabla_I^2 A}{A} , $$

$$ \frac{\partial A}{\partial t} + \sum_I \frac{1}{M_I} (\nabla_I A)
    (\nabla_I S) + \sum_I \frac{1}{2M_I} A (\nabla_I^2 S) = 0 . $$

This set of equations is known as the "quantum fluid dynamical representation".
The term for $S$ contains a term for $\hbar$ which vanishes in
the classical limit $\hbar \rightarrow 0$:

$$ \frac{\partial S}{\partial t} + \sum_I \frac{1}{2M_I}
    (\nabla_I S)^2 + \int d\bm{r} \Psi^* \hat{H} \Psi = 0 . $$

This formulation of the nuclear dynamics is isomorphic
to the Hamilton-Jacobi formulation:

$$ \frac{\partial S}{\partial t} + \hat{H} = 0 , $$

with the classical Hamilton function

$$ \hat{H} = T(P_I) + V(R_I) , $$

with coordinates $R_I$ and conjugate momenta $P_I$.
\par
If we identify the conjugate momenta

$$ \bm{P}_I = \nabla_I S , $$

we obtain the Newtonian equations of motion:

\begin{equation}
    \begin{split}
        \frac{d\bm{P}}{dt} 
    &= -\nabla_I V
    = -\nabla_I \int d\bm{r} \Psi^* \hat{H} \Psi \quad \text{or} \\
        M_I\frac{d^2 \bm{R}_I}{dt^2}
    &= -\nabla_I \int d\bm{r} \Psi^* \hat{H} \Psi \\
    &= -\nabla_I V_e^E (R_I(t)) .
    \end{split}
\end{equation}

The nuclei now move according to classical mechanics
in an effective potential $V_e^E$ generated by the electrons.
This potential is a function only of the nuclear
degrees of freedom at time $t$ after averaging out
the electronic degrees of freedom.
\par
For consistency the nuclear wavefunction appearing
in the TDSCF equation for the electronic
degrees of freedom has to be replaced by the positions
of the nuclei point particles.
This is done by replacing the nuclear density $\left| \chi \right|^2$
in the limit $\hbar \rightarrow 0$ by a product of delta functions
$ \prod_I \delta (\bm{R}_I - \bm{R}_I(t)) $ centered
at the instantaneous positions $\set{\bm{R}_I(t)}$
of the classical nuclei.
This leads to a time-dependent wave equation
for the electrons:

$$ i\hbar\frac{\partial \Psi}{\partial t} =
    -\sum_i \frac{\hbar}{2m_e} \nabla_i^2 \Psi
    + V_{ne} \Psi , $$

which evolve quantum mechanically as the nuclei propagate
classically.
This mixed approach is commonly referred to as
\textit{Ehrenfest molecular dynamics}.
Although the underlying equations describe a mean-field
theory, the Ehrenfest approach includes transitions
between electronic states.
\par
To arrive at a purely classical description of the
dynamics of both the nuclei and the electrons
we need to make further simplifications. \\
Firstly we restrict the electronic wave function $\Psi$
to the ground state wave function $\Psi_0$
at every instant of time.
This means the nuclei move on a single potential energy surface

$$ V_e^E = \int d\bm{r} \Psi_0^* \hat{H} \Psi_0 = E_0(R_I) , $$

that is determined by solving the Schrodinger equation

$$ \hat{H} \Psi_0 = E_0 \Psi_0 . $$

In this limit the Ehrenfest potential is identical
to the ground state Born-Oppenheimer potential.

\subsection{Molecular dynamics simulations}

\subsection{Molecular dynamics potentials}
