In this thesis we have trained neural networks on atomic configurations
to predict potential energy and forces. Using the Behler-Parrinello
method to map atomic configurations to a one-dimensional feature vector,
we preserve the required symmetries of translation, rotation and permutation
while mapping the configurations to a form which can be efficiently
processed by a neural network.
Using the Atomistic Machine-learning Package in conjunction
with the Atomic Simulation Environment we trained neural networks
to reproduce a system of copper atoms governed by the Effective Medium
Theory Potential and a system of silicon atoms governed
by the Stillinger-Weber potential.
We find large differences in mechanical properties and numerical stability
from these potentials, these differences are likely caused
by the difference in the average number of neighbors, the number
of and type of symmetry functions applied and the functional form
of the potentials.
However, while the neural network can more accurately reproduce
the Stillinger-Weber potential, we find in both cases that the
energy increases over time, along with an increase in translational
momentum.

\subsection{Properties of the neural network potential}

\subsection{Prospects and future work}

\subsubsection{Energy conservation}

\subsubsection{Improved sampling}

\subsubsection{Ab-initio molecular dynamics}

\subsection{Alternative fingerprinting schemes}

\subsubsection{Optimization}
