This chapter will give a brief overview of the Hartree-Fock
and Density Functional Theory methods, which will be our primary workhorses
for generating quantum mechanical trajectories.
\par
Something something Born-Oppenheimer nucleonic degrees of freedom.

% More intro?

\subsection{Hartree-Fock}
The theory in this section is adapted from the lectures notes
for the course FYS4480 - Many-body Physics at the University of Oslo [ref].
The Hartree-Fock method is an approximate method for solving
a many-body system of fermions, with a Hamiltonian 
which can be written as the sum of a onebody part and a twobody interaction:

$$ \hat{H} = \hat{H}_1 + \hat{H}_2 =
\sum_i \hat{h}_1 (\bm{r}_i) + \sum_{i < j} \hat{v}(r_{ij}) . $$

The onebody part is typically chosen as a potential for which
we have an exact solution, such as the harmonic oscillator
or the coulomb attraction of an electron to the nucleus.
\par
Hartree-Fock is a method for finding the best possible
approximation to the ground state wavefunction $\Psi$
assuming it can be written as a \textit{Slater determinant} $\Phi$
of orthonormal spin orbitals $\psi$:

\begin{equation}
\Phi(\bm{r}_1,\dots,\bm{r}_N, \alpha, \beta, \dots, \sigma)
= \frac{1}{\sqrt{N}}
\begin{vmatrix}
 & x_{0} & x_{0}^{2} & \dots & x_{0}^{n} \\ 
1 & x_{1} & x_{1}^{2} & \dots & x_{1}^{n} \\
\hdotsfor{5} \\
1 & x_{n} & x_{n}^{2} & \dots & x_{n}^{n}
\end{vmatrix}
\end{equation}
