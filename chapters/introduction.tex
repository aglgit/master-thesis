In recent years the field of artificial intelligence has
made great strides, particularly in applications such
as computer vision, natural language processing,
speech recognition and user behaviour analytics.
The exponential growth of available and high quality
data has brough about a large demand for accurate, efficient
and semi-automated algorithms, capable of building complicated
mathematical models and making predictions without human supervision.
Artificial neural networks in particular, partially inspired by simple
models of human cognition, have found success in a large variety
of task, largely owing to their ability to scale efficiently
as the size of the dataset increases.
\par
Data analysis and machine learning have become important parts
of many subfields of physics, such as observational astronomy,
condensed matter, biophysics and subatomic particle physics.
Electronic structure calculations are well suited for the
adoption of machine learning methods, as large datasets
can be produced on demand and relatively free of noise which
obscures patterns.

\subsection{Electronic structure calculations}
Electronic structure calculations - or numerical solutions
to the manybody Schr\"{o}dinger equation - are methods of modelling
physical systems based on the principles of quantum mechanics.
They involve very few parameters, and the results are found
to be very accurate compared to laboratory measurements.
However, exponential scaling greatly limits their applications, 
and we are typically limited to very small and simple systems.
Approximations can be made which reduces the scaling
to polynomial, but this limits the accuracy of the method.
The most succesful and widely applied methods
are the Density Functional Theory (DFT) method
and the Hartree-Fock (HF) family of methods.
The Hartree-Fock scales nominally as $\mathcal{O}(N^4)$,
where $N$ is the number of particles in the system
(i.e. electrons), but typically as $\mathcal{O}(N^3)$.
The Density Functional Theory method scales in a similar
manner, but with a larger proportionality term.
However, Hartree-Fock requires us to keep track of
the coordinates and spin of every particle, while
DFT offers similar accuracy and scaling
and is a function of only three spatial coordinates.
\par
Molecular Dynamics (MD) freezes out the degrees of freedom of the
electrons, and treats the atoms as point particles centered on
the nucleus using a mean-field approach.
The dynamics of the particles are treated as Newtonian,
meaning the position and velocity can be integrated using a
symplectic integrator such as Verlet integration.
This treatment allows for systems much larger
than ab-initio methods, but at the cost of neglecting
both nuclear and electronic phenomena.
However, the development of molecular dynamics potentials
involves a large amount of parameters for any realistic system,
and determining their functional form is more of an
art than an exact science.
\par
An application of machine learning which has shown some promise
is the development of molecular dynamics potentials
or Potential Energy Surfaces (PES). Using for example
artificial neural networks which are universal function approximators
the parameters are fully determined from the training data.
This offers the potential of developing potential energy surfaces
directly from ab-initio data or from well-developed and tested
empirical potentials.

\subsection{Atom-centered descriptors}
For electronic structure calculations the potential energy,
forces and other derived properties are determined by
the cartesian coordinates. However, it is not sufficient
to feed neural networks cartesian coordinates labeled with
the energy. From physical theory and intuition any molecular dynamics
potential must respect all of translation, rotation and permutation invariance.
In order to apply machine learning methods we require a mapping from
the cartesian coordinates to a 1D feature vector which conserves
all of these properties.
\par
The idea is to approximate the potential energy with a regression model:

\begin{equation}
\left\{\bm{R\right\}} \overset{\text{regression}}{\longrightarrow} 
E = E\left( \left\{\bm{R\right\}} \right) ,
\end{equation}

where $\left\{\bm{R\right\}}$ represents the cartesian coordinates of our system.
A mapping that satisfies the constraints of translation, rotation
and permutation invariance we will refer to as a \textit{descriptor}.
The descriptor is a multidimensional function $\bm{G}$
that serves as input to the regression method:

\begin{equation}
\left\{\bm{R\right\}} \rightarrow \bm{G} \left(\left\{\bm{R\right\}} \right)
\overset{\text{regression}}{\longrightarrow}
E = E \left(\bm{G}\left(\left\{\bm{R\right\}}\right)\right)
\end{equation}

Once we have a descriptor and a regression model the dynamics
can be readily obtained by taking derivatives:

\begin{equation}
\begin{split}
    \bm{F}_i &= -\nabla_i E \\
    &= -\nabla_i
    E(\bm{G}(\left\{\bm{R\right\}})) \\
    &= -\sum_j \frac{\partial E}
    {\partial G_j} \frac{\partial G_j}{\partial \bm{R}_i} ,
\end{split}
\end{equation}

where we have applied the chain rule to break the gradient
into derivatives with respect to the network inputs (obtained through
backpropagation) and derivatives of the network inputs with
respect to the coordinates of atom $i$.
The potential energy is typically also broken into
atomic contributions:

\begin{equation}
E = \sum_{i=1}^N E_{\text{atom}}\left(\bm{R}_i, \left\{\bm{R\right\}} \right) ,
\end{equation}

where each atomic contribution is determined by the atom's
local environment. This allows us to treat systems with
a varying amount of particles without retraining
the machine learning method each time.

\subsection{Goals}

\subsection{Contributions}

\subsection{Structure}
