The discussion in this chapter follows closely the discussion in
[Sakurai | Modern Quantum Mechanics].
\newline
In quantum mechanics, a physical state is represented by a \textit{state vector}
in a complex vector space. Such a vector is called a \textit{ket}, denoted
by $\ket{\alpha}$. The state ket is postulated to contain all information
about the physical state. Two kets can be added to produce a new ket:
$$ \ket{\alpha} + \ket{\beta} = \ket{\gamma} .$$
They can also be multiplied by a complex number:
$$ c\ket{\alpha} = \ket{\alpha}c = \ket{\delta} .$$
If $c$ is zero the resulting ket is called a \textit{null ket}.
If $c$ is non-zero it is postulated that the resulting ket contains
the same information.
\newline
Observables such as momentum and spin are represented by operators
acting on the vector space in question. Operators
act on a ket from the left to produce a new ket:
$$ A \ket{\alpha} = \ket{\delta} .$$
Of particular importance is when the action of an operator
on a ket is the same as multiplication:
$$ A \ket{\alpha} = c\ket{\alpha} = \ket{\delta} .$$
These kets are known as \textit{eigenkets} and the corresponding
complex numbers are known as \textit{eigenvalues}.
The physical state represented by an eigenket is known
as an \textit{eigenstate}. Any ket
can be written as an expansion of eigenkets $\ket{a'}$:
$$ \ket{\alpha} = \sum_{a'} c_{a'} \ket{a'} ,$$
where $c_{a'}$ is a complex coefficient. In principle
there are infinitely many linearly indepedent eigenkets,
depending on the dimensionality of the vector space.
The uniqueness of the expansion can be proven
with orthonogality of the eigenkets, which we will simply postulate.
\par
A \textit{bra space} is a vector space "dual" to the ket space.
We postulate that for every ket $\ket{\alpha}$ there exists a bra
$\bra{\alpha}$. The bra space is spanned by eigenbras $\bra{a'}$
corresponding to the eigenkets $\ket{a'}$. The ket and bra spaces
have a dual correspondence:
$$ \ket{\alpha} \leftrightarrow \bra{\alpha} $$
$$ \ket{\alpha '}, \ket{\alpha ''},\dots \leftrightarrow \bra{\alpha '}, \bra{\alpha ''},\dots $$
$$ \ket{\alpha} + \ket{\beta} \leftrightarrow \bra{\alpha} + \bra{\beta} .$$
The bra dual to $c \ket{\alpha}$ is postulated to be $c^* \ket{\alpha}$,
and more generally:
$$ c_{\alpha} \ket{\alpha} + c_{\beta} \ket{\beta} \leftrightarrow
c_{\alpha}^* \bra{\alpha} + c_{\beta}^* \bra{\beta} .$$
The \textit{inner product} of a bra and a ket is a complex number
written as a bra on the left and a ket on the right.
It has the fundamental property:
$$ \braket{\alpha | \beta} = \braket{\beta | \alpha}^* ,$$
in other words they are complex conjugates.
For this to satisfy the requirements of an inner product we must have
$$ \braket{\alpha | \alpha} \geq 0 ,$$
with equality if and only if $\ket{\alpha}$ is a null ket.
We define the \textit{norm} of a ket as
$$ \sqrt{\braket{\alpha | \alpha}} ,$$
which can be used to form normalized kets
$$ \ket{\alpha^{\sim}} \frac{1}{\sqrt{\braket{\alpha | \alpha}}} \ket{\alpha} ,$$
with the property
$$ \braket{\alpha^{\sim} | \alpha^{\sim}} = 1 .$$
Two kets are said to be \textit{orthogonal} if
$$ \braket{\alpha | \beta} = 0 .$$

\subsection{Operators}
As we briefly mentioned above, operators act on kets from the left
to produce a new ket. Two operators $A$ and $B$ are equal
$$ A = B $$
if
$$ A \ket{\alpha} = B \ket{\alpha} $$
for an arbitrary ket in the relevant ket space. An operator $A$
is said to be the \textit{null operator} if
$$ A \ket{\alpha} = 0 .$$
Operators can be added, and addition operations are commutative and associative.
$$ X + Y = Y + X ,$$
$$ (X + Y) + Z = X + (Y + Z) .$$
\par
Operators act on bras from the right to produce a new bra
$$ \bra{\alpha} A = \bra{\beta} .$$
The ket $A \ket{\alpha}$ and the bra $\bra{\alpha} A$ are in general
not dual to each other. We define the \textit{hermitian adjoint} $A^{\dagger}$ as
$$ A \ket{\alpha} \leftrightarrow \bra{\alpha} A^{\dagger} .$$
An operator is said to be \textit{hermitian} if
$$ A = A^{\dagger} .$$
Operators can be multiplied.
Multiplication is associative, but non-commutative:
$$ XY \neq YX ,$$
$$ X(YZ) = (XY)Z .$$
The left product of a ket and a bra is known as the \textit{outer product}:
$$ \ket{\alpha} \bra{\beta} .$$
The outer product should be treated as an operator, while the inner product
$\braket{\alpha | \beta}$ is a complex number.
\newline
If an operator is to the left of a ket $\ket{\alpha} A$ or to the right
of a bra $A \bra{\beta}$ these are illegal products, in other words
not defined within the ruleset of quantum mechanics.
The associative properties of operators are postulated to hold true
as long as we are dealing with legal multiplications among kets, bras
and operators. As an example, the outer product acting on a ket:
$$ (\ket{\alpha} \bra{\beta}) \ket{\gamma} ,$$
can be equivalently regarded as scalar multiplication
$$ \ket{\alpha} (\braket{\alpha | \gamma})
    = \ket{\alpha} c = c \ket{\alpha} ,$$
where $c = \braket{\alpha | \gamma}$ is just a complex number.

% probably needs some stuff to derive Schrodinger
% probably needs some stuff to derive Schrodinger
% probably needs some stuff to derive Schrodinger

\subsection{Time evolution}
In quantum mechanics, time is treated not as an observable,
but as a parameter. Relativistic quantum mechanics
treats space and time on the same footing, but only by demoting
position to a parameter.
\newline
Suppose we have a physical system $\ket{\alpha}$
at a time $t_0$. Denote the ket at a later time $t > t_0$ by
$$ \ket{\alpha, t; t_0} .$$
As time is assumed to be a continuous parameter we expect
as we evolve the system backward in time
$$ \lim_{t \rightarrow t_0} \ket{\alpha, t; t_0}
= \ket{\alpha} .$$
The kets separated by a time $t - t_0$
are related by the \textit{time-evolution operator} $\mathcal{U}$:
$$ \ket{\alpha, t; t_0} = \mathcal{U}(t, t_0) \ket{\alpha, t_0} .$$
If the state ket is normalized to unity at a time $t_0$,
it must remain normalized at a later time:
$$ \braket{\alpha, t_0 | \alpha, t_0} = \braket{\alpha, t; t_0
    | \alpha, t; t_0} = 1 .$$
This is guaranteed if the time evolution operator 
$\mathcal{U}$ is a \textit{unitary} operator:
$$ \mathcal{U}^{\dagger} \mathcal{U} = 1 .$$
We also require the composition property:
$$ \mathcal{U}(t_2, t_0) = \mathcal{U}(t_2, t_1)
    \mathcal{U}(t_1, t_0), ~ (t_2 > t_1 > t_0) .$$
\newline
If we consider an infinitesimal time-evolution operator
$$ \ket{\alpha, t_0 + dt; t_0} = \mathcal{U}(t_0 + dt, t_0)
\ket{\alpha, t_0} ,$$
it must reduce to the identity operator as the infinitesimal
time interval $dt$ goes to zero:
$$ \lim_{dt \rightarrow 0} \mathcal{U}(t_0 + dt, t_0) = 1, $$
and we expect the difference between the operators
to be of first order in $dt$.
\newline
These requirements are all satisfied by
$$ \mathcal{U}(t_0 + dt, t_0) = 1 - i\Omega dt ,$$
where $\Omega$ is a Hermitian operator:
$$ \Omega^{\dagger} = \Omega .$$
The operator $\Omega$ has the dimension inverse time.
Frequency or inverse time is related to energy
through the Planck-Einstein relation:
$$ E = \hbar \omega .$$
In classical mechanics the Hamiltonian is the generator of time evolution,
so we postulate that t $\Omega$ is related to the Hamiltonian operator
$H$:
$$ \Omega = \frac{H}{\hbar} .$$
The Hamiltonian operator represents the energy of our system,
which is a physical observable and must therefore be Hermitian.

\subsection{The Schrodinger equation}
By exploiting the composition property of the time-evolution
operator we obtain:
$$ \mathcal{U}(t + dt, t_0) = \mathcal{U}(t + dt, t)
    \mathcal{U}(t, t_0) = (1 - \frac{i H dt}{\hbar})
    \mathcal{U}(t, t_0) ,$$
where the time difference $t - t_0$ is not required to be infinitesimal.
This implies
$$ \mathcal{U}(t + dt, t_0) - \mathcal{U}(t, t_0) =
    -i(\frac{H}{\hbar}) dt \mathcal{U}(t, t_0) ,$$
and taking the limit $dt \rightarrow 0$:
$$ i \hbar \frac{\partial}{\partial dt} \mathcal{U}(t, t_0)
    = H \mathcal{U}(t, t_0) .$$
This is known as the Schrodinger equation for the time-evolution operator.
Multiplying both sides by a ket $\ket{\alpha, t_0}$ leads to the
Schrodinger equation:
$$ i \hbar \frac{\partial}{\partial dt} \mathcal{U}(t, t_0)
    \ket{\alpha, t_0} = H \mathcal{U}(t, t_0) \ket{\alpha, t_0} .$$
As the ket does not depend on time this is the same as
$$ i \hbar \frac{\partial}{\partial dt}
    \ket{\alpha, t_0} = H \ket{\alpha, t_0} .$$
