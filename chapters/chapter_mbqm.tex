This chapter will give a brief overview of the Hartree-Fock
and Density Functional Theory methods, which are the primary workhorses
in manybody quantum theory.
\par
In most applications of quantum theory we are interested in
finding solutions to the non-relativistic time-independent
Schrodinger equation:

$$ \hat{H} \ket{\Psi} = E \ket{\Psi} ,$$

with the Hamiltonian $\hat{H}$ describing a system of nuclei and electrons
with coordinates $\bm{R}_i, \ i=1,2,\dots,A$ and $\bm{r}_i, \
i=1,2,\dots,N$ respectively. The distance between nuclei $a$
and electron $i$ is given as $r_{ai} = \left| \bm{R}_a - \bm{r}_i \right|$
and correspondingly for the nuclei-nuclei and electron-electron distances.
\par
The full Hamiltonian for a set of $N$ electrons and $A$ nuclei
in atomic units is

\begin{equation}
    \begin{split}
        \hat{H} 
        &= -\sum_{i=1}^N \frac{1}{2} \nabla_i^2
        -\sum_{a=1}^A \frac{1}{M_a} \nabla_a^2
        -\sum_{i=1}^N \sum_{a=1}^A \frac{Z_a}{r_{ia}} \\
        &+ \sum_{i=1}^N \sum_{j=i+1}^N \frac{1}{r_{ij}}
        + \sum_{a=1}^A \sum_{b=a+1}^A \frac{Z_a Z_b}{R_{ab}}
    \end{split} .
\end{equation}

The first two terms describe the kinetic energy operators
of the electrons and nuclei, with $M_a$ the ratio of the mass
of nuclei $a$ to the electron mass. The third term describes the
coulomb attraction between electron and nuclei, while the fourth and fifth
terms describe the repulsion between electrons and nuclei respectively.
\par
From this description we are most often interested in the electronic
structure problem presented by applying the \textit{Born-Oppenheimer}
approximation. Since the nuclei are approximately 2000 times heavier
than the electrons, the electrons can to a good approximation
be described as moving in the field of fixed nuclei. This means we
can neglect the kinetic energy terms of the nuclei, while considering
an averaged effect from the nuclei-nuclei repulsion.
The nuclei-nuclei repulsion energy adds a constant to the energy
eigenvalues, but has no effect on the energy eigenfunctions.
The remaining terms are known as the electronic Hamiltonian:

\begin{equation}
    \hat{H}_e = -\sum_{i=1}^N \frac{1}{2} \nabla_i^2
    -\sum_{i=1}^N \sum_{a=1}^A \frac{Z_A}{r_{ia}}
    +\sum_{i=1}^N \sum_{j=i+1}^N \frac{1}{r_{ij}} .
\end{equation}

The electronic wavefunction $\Psi_e = \Psi_e(\{r_i\}; \{R_a\})$
is a function of the electronic coordinates with a parametric dependence
on the fixed nucleic coordinates. We can also choose to neglect
the relative positions of the nuclei and consider only the
center of mass of the nucleus with a total charge $Z$,
which gives us the final expression we want for our Hamiltonian:

\begin{equation}
    \hat{H} = -\sum_{i=1}^N \frac{1}{2} \nabla_i^2
    - \sum_{i=1}^N \frac{Z}{r_{i}} + \sum_{i=1}^N \sum_{j=i+1}^N
    \frac{1}{r_{ij}}
\end{equation}

\subsection{Hartree-Fock}
The theory in this section is adapted from the lectures notes
for the course FYS4480 - Many-body Physics at the University of Oslo [ref].
The Hartree-Fock method is an approximate method for solving
a many-body system of fermions, with a Hamiltonian 
which can be written as the sum of a onebody part and a twobody interaction:

$$ \hat{H} = \hat{H}_1 + \hat{H}_2 =
\sum_i \hat{h}_1 (\bm{r}_i) + \sum_{i < j} \hat{v}(r_{ij}) . $$

The onebody part is typically chosen as a potential for which
we have an exact solution, such as the harmonic oscillator
or the coulomb attraction of an electron to the nucleus.
For the electronic structure problem these take the form of:

\begin{equation}
    \hat{H}_1 = -\sum_{i=1}^N \frac{1}{2} \nabla_i^2
    - \sum_{i=1}^N \frac{Z}{r_{i}} 
\end{equation}

\begin{equation}
    \hat{H}_2 =
    \sum_{i=1}^N \sum_{j=i+1}^N \frac{1}{r_{ij}}
\end{equation}

Hartree-Fock is a method for finding the best possible
approximation to the ground state wavefunction $\Psi$
assuming it can be written as a \textit{Slater determinant} $\Phi$
of orthonormal spin orbitals $\phi$:

\begin{equation}
\Phi(\bm{r}_1,\dots,\bm{r}_N, \alpha, \beta, \dots, \sigma)
= \frac{1}{\sqrt{N}}
\begin{vmatrix}
    \phi_{\alpha}(\bm{r}_1) & \phi_{\alpha}(\bm{r}_2) 
    & \dots & \phi_{\alpha}(\bm{r}_N) \\
    \phi_{\beta}(\bm{r}_1)  & \phi_{\beta}(\bm{r}_2)
    & \dots & \phi_{\beta}(\bm{r}_N) \\
    \hdotsfor{4} \\
    \phi_{\sigma}(\bm{r}_1) & \phi_{\sigma}(\bm{r}_2) 
    & \dots & \phi_{\sigma}(\bm{r}_N)
\end{vmatrix} ,
\end{equation}

with particle coordinates $\bm{r}_i, \ i=1,2,\dots,N$
and $\alpha, \beta, \dots \sigma$ quantum numbers
needed to describe the quantum state of the particles.
The spin orbitals $\phi$ are solutions to the onebody
Hamiltonian:

\begin{equation}
    \hat{H}_1 \phi_{\alpha} = \epsilon_{\alpha} \phi_{\alpha} .
\end{equation}



