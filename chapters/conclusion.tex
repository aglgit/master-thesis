In this thesis we have trained neural networks to reproduce
molecular dynamics potentials using the Behler-Parrinello method.
These potentials are high-dimensional functions with many parameters
to be determined, and are often developed through intimate knowledge
of the physical and chemical properties of the systems they are designed for.
In addition there are certain symmetries which must be respected when
developing potentials, in particular translational, rotational and permutational
symmetries, as well as conserving energy over time (in the NVE ensemble).
Traditional ab-initio methods suffer from poor scaling as the size of the systems
increases, while classical potentials derived from ab-initio calculations
allows us to simulate realistic scales of up to millions of atoms
depending on the computer resources and the complexity of the atoms in the system.
Since these neural networks when trained offer linear scaling,
they could serve both as supplements to ab-initio methods
or be deployed for calculations using classical molecular dynamics.
For example, one could save many CPU cycles by producing and training
on a small trajectory calculated from DFT, and then use the neural networks
to produce the remainder of the data, provided the neural networks
are reasonably accurate.
While our neural network potential scales linearly with system size,
it has a rather large pre-factor dependent on the symmetry function set,
cutoff radius and the average number of neighbors in the system.
We believe this could be improved by moving more calculations to lower
level compiled languages, such as calculations of neighbor lists,
fingerprints and fingerprint derivatives for every step.
Although we have only ran the neural network on a single core,
the potential could be parallelized over the atoms
using algorithms such as LAMMPS neighbor lists without
too much overhead.
\par
Unfortunately we were not able to achieve energy conservation
with out neural network. This a central feature of any neural network potential,
and without energy conservation, we cannot sample properties in equilibrium,
as for example the radial distribution function or the diffusion constant
is increasing over time and ill defined.
Generally we find an increase in kinetic energy over time, corresponding
to an increase in potential energy as the atoms move apart,
and an increase in translational momentum.
The results are notably different for the EMT and Stillinger-Weber potential,
this is caused by among other things the symmetry function sets and the
average number of neigbors in the system.
The Stillinger-Weber explicitly includes threebody interactions,
while the EMT potential is a function of interatomic distances, and our
results suggest that the balance between the number of radial and symmetry functions
should be decided by more careful analysis of their relative importance.
\par
On the test trajectory we achieved reasonable values of the energy and force
root mean squared errors, approximately 0.1 eV for the energy and 0.05-0.1 eV/Å
for the force. These are mostly consistent with the results others have achieved,
though the energy is often on the order of 1-10 meV, which is lower than
what we have achieved. This is partly due to the fact that we have emphasized
training with forces, which comes at the expense of the energy fit.
Since the training RMSE is substantially lower than the test RMSE this may
indicate overfitting, though we would not expect this to be a substantial problem
with such a small network. We note that this has not been tested extensively,
and more testing could produce different results.

\subsection{Prospects and future work}
In this thesis we barely scratched the surface of combining machine learning
methods with molecular dynamics. There are therefore many, many paths
to be explored for future theses or even articles to be published,
which may be achieved working through ASE and/or AMP and making modifications
or writing your own code from scratch. We will list some of the prospects
considered while writing this thesis in semi-ordered order of importance,
though there are likely many which have not been considered.
TODO: Fill out list.

\begin{itemize}
    \item Numerical optimization
    \item Determining symmetry function sets
    \item Improved sampling methods
    \item Finding minima
    \item New descriptors
    \item New machine-learning models
    \item Multiple atoms
    \item Long-range interactions
    \item Ensembles
\end{itemize}

