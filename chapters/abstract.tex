Artificial neural networks are fitted to molecular dynamics trajectories
using the Behler-Parrinello method of atom-centered symmetry functions
in order to obtain analytical interatomic potentials.
Molecular dynamics trajectories are generated using the Atomic Simulation
Environment (ASE) and the neural networks are initialized and trained
using the Atomistic Machine-Learning Package (AMP). AMP is interfaced
with ASE through the Calculator interface, which is a black box
that accepts atomic numbers and atomic positions and calculates
the energy and, if implemented, forces and stresses.
\par
Neural network potentials are constructed for Copper and Silicon
in equilibrium crystal structures, and are evaluated on the potential energy,
energy conservation, radial distribution function and mean
squared displacement, as well as the absolute errors of the
potential energies and force components on the test trajectories.
We find the neural networks are able to reproduce the crystal structures,
but obtain negative results for the ability to conserve energy,
leading to an increase in kinetic energy and translational momentum
over time, with negative implications for long-term numerical stability.
Recommendations for future work include better sampling
algorithms for sampling likely configurations out of equilibrium,
testing different numerical optimization algorithms
and a more efficient implementation of the Behler-Parrinello
symmetry functions for facilitating faster training and deployment
of different architectures on available training data, as well as
on new input data.
